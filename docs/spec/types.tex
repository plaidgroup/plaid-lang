\section{Types}

\begin{quote}

\ntermdef{Type}

\defspace \keyw{void}

\defspace \opt{\keyw{immutable}} \nterm{LambdaStructure}

\defspace \opt{\nterm{Permission}} \nterm{NominalStructure}

\defspace (\nterm{Type})

\ntermdef{Permission}

\defspace \keyw{unique}

\defspace \nterm{SymmetricPermission}

\defspace \nterm{LocalPermission}

\ntermdef{SymmetricPermission}

\defspace \keyw{shared} <\nterm{SimpleExpr1}>

\defspace \keyw{immutable}

\ntermdef{LocalPermission} 

\defspace \keyw{local} \nterm{SymmetricPermission}

\ntermdef{LambdaStructure} 

\defspace \opt{\nterm{MetaParams}} (\nterm{ArgSpecs}) \opt{[ \nterm{Args}]} -> \nterm{Type}

\ntermdef{NominalStructure} 

\defspace \keyw{top}

\defspace \nterm{QualifiedIdentifier} \opt{MetaArgs}

\ntermdef{ArgSpecs}

{\defspace \nterm{ArgSpec} \seq{ * \nterm{ArgSpec} }}



\end{quote}

\keyw{void} is the most general type in Plaid.  It represents the weakest
and unwritable permission \keyw{none} with structure \keyw{top}
which gives no information about the object's abilities.  Parameters
and some variables without type annotations are given the
type \keyw{dynamic}.  Values of type \keyw{dynamic} must
be cast to another type to be used by the type system.  All other
types include an optional \nterm{permission} and a structure.

\nterm{LambdaStructures} represent function types.  They optionally
include the initial and resulting types of references in scope during a
function call list in [ ] as in function and method declarations. 
Formally, a function that accepts multiple arguments actually accepts an
argument tuple, which is written with a \code{*}-separated list.

\nterm{NominalStructures} represent the types of declared states
and the special \keyw{top} structure which is a superstructure
of all structures.

If the permission for a structure is not given, then a default is applied.
\nterm{LambdaStructures} can only be declared \keyw{immutable}
and also default to \keyw{immutable}.  The \keyw{top} structure 
defaults to \keyw{none} and a \nterm{NominalStructures} defaults 
to \keyw{immutable} if it represents an \keyw{immutable} state 
and \keyw{unique} otherwise. 

The \keyw{unique} permission indicates that there are no
usable aliases to the same object.  There may be other
references to the object with the \keyw{none} permission
which does not allow the object to be used in any way.

\nterm{SymmetricPermissions} allow new aliases to be 
created with the same permission.
\keyw{immutable} references cannot update the object
but can assume that it never changes. \keyw{shared} references
can make changes, but must assume that other
references may have changed the object.

\keyw{local} permissions give the same abilities
and guarantees as their underlying \nterm{SymmetricPermission}, 
but are restricted to local variables.  
\keyw{local} references cannot be assigned into fields.
This restriction allows \keyw{local} permissions to be
returned to their original location to regain a stronger permission.
For example, a \keyw{unique} reference passed to a function
that requires and results in a \keyw{local} permission will
still be \keyw{unique} after the function call.

