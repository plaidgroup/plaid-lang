\documentclass[12pt]{article}
\usepackage{fullpage,cmu-titlepage2}
\usepackage{times}
\usepackage{excludeonly}

\usepackage[T1]{fontenc}

\usepackage{amssymb}
\usepackage{listings}
\usepackage{mathpartir}
\usepackage{graphicx}
\usepackage{tabularx}
\usepackage[release]{PlaidDefinitions}
\usepackage{datetime}
\usepackage{natbib}

\newcommand{\singlespace}{\renewcommand{\baselinestretch}{1.0}\normalsize}

\renewcommand{\cite}{\citep}


%%%%%%%%%%%%%%%%%%%%%%%%%%%%%%%%%%%%%%%%%%%%%%%%%%%%%%%%%%%%%%%%%%%%%%%%%%%%%

\title{The Plaid Language:\\
Typed Core Specification\\
\vspace{2ex}
version 0.4.0\\
\vspace{2ex}
}

\author{Jonathan Aldrich \and Nels E. Beckman \and Robert Bocchino \and Karl Naden \and Darpan Saini \and Sven Stork \and Joshua Sunshine}


\date{\monthname~\the \year}

\abstract{
Plaid is an object oriented programming language built on two paradigms.  First, Plaid
is {\it typestate-oriented}. Programmers can directly encode the {\it abstract states}
of objects and use the {\it state change operator} to change the state, interface, and representation
of an object at runtime.  Second, Plaid's type system is {\it permission-based}.  The type of each reference 
includes an {\it access permission} which dictates how the reference can be used and characterizes
the permissions to other aliases of the same object.  Plaid leverages permissions 
when tracking the abstract state of references during typechecking.  Permissions are
also used to infer code that can be safely run in parallel.
This document defines the core of the Plaid language, including its source syntax, 
the semantics of operations involving abstract states, and a type system.
}





\keywords{programming language, typestate, Plaid, gradual typing, permissions}

\trnumber{CMU-ISR-12-103}


\support{This research was supported by DARPA grant HR00110710019, CMU|Portugal Aeminium grant CMU-PT/SE/0038/2008, NSF grants CCF-0811592 and CCF-1116907, and grant \#1019343 to the Computing Research Association for the
CIFellows Project.}


\begin{document}

\renewcommand*{\thepage}{title-\arabic{page}} 
\maketitle
\renewcommand*{\thepage}{\arabic{page}} 




%%%%%%%%%%%%%%%%%%%% MAIN TEXT %%%%%%%%%%%%%%%%%%%%%%%%%%%%%%

\section{Conventions}

This document uses the same grammar definition conventions as the Java
Language Specification, Third Edition (JLS) ~\cite{gosling2005}.  Those conventions
are described in chapter 2 of the JLS and are not repeated here.




\section{Lexical Structure}

The lexical structure of Plaid is based on that of Java, as defined in
chapter 3 of the JLS.  Specifically:

\begin{itemize}

\pII{\item Plaid uses Unicode, as does Java (JLS section 3.1).}

\pII{\item Plaid supports the same lexical translations as Java (JLS section 3.2) and the same Unicode escapes (JLS section 3.3).}

\item Plaid uses the same definitions of line terminators as Java (JLS
  section 3.4), the same input elements and tokens (JLS section 3.5)
  except for a different keyword list, and the same definition of
  whitespace (JLS section 3.6).

\item Plaid uses the same definition of comments as Java (JLS section 3.7).

\item Plaid uses the same definition of identifiers as Java (JLS section 3.8).

\item Plaid uses the same definition of literals as Java (JLS section
  3.10) except that there is no Null Literal.  Furthermore, Boolean
  objects named \texttt{true} and \texttt{false} exist in the standard
  library, but unlike in Java these are not keywords in Plaid.
  \pI{Version 1.0 of Plaid supports only String literals, plus a
    restricted form of Integer literal which are strings of digits.}

\item Plaid uses the same definition of Separators as Java (JLS section 3.11).

\end{itemize}


\subsection{Keywords}

The following character sequences, formed from ASCII letters, are reserved
for use as \textit{keywords} and cannot be used as identifiers:

\begin{quote}
\ntermdef{Keyword} \oneof 
\end{quote}
\[
  \begin{array}{cccccc}
  \keyw{case} 
  & \keyw{default}
  & \keyw{import}
  & \keyw{fn}
  & \keyw{match}
  \\
  \keyw{method}
  & \keyw{new}
  & \keyw{of}
  & \keyw{overrides}
  & \keyw{package}
  \\
  \keyw{requires}
  & \keyw{state}
  & \keyw{this}
  & \keyw{val}
  \\
  \keyw{var}
  & \keyw{with}
  \\
  \vII{\keyw{dyn} & \keyw{dynamic} & \keyw{immutable} & \keyw{unique} & \keyw{none}}
  \end{array}
\]

%\TODO{The list above is only a current estimate, this will change as
%  we get the whole language defined.  Should also put the above into
%a nice table as in the JLS.}

%Note that \keyw{true} and \keyw{false} are actually boolean literals,
%as in Java.





\subsection{Operators}

We first define operator characters as follows:

\begin{quote}

\ntermdef{OperatorChar} \oneof

\defspace \texttt{= < > ! $\sim$ ? : \& | + - * / \^{} \%}

\end{quote}

Now an operator is a sequence of operator characters:

\begin{quote}

\ntermdef{Operator}

\defspace \nterm{OperatorChar}

\defspace \nterm{OperatorChar} \nterm{Operator}

\end{quote}

The exception to the grammar above is that the character sequences
\texttt{=}, \texttt{=>} and \texttt{<-} have other meanings in the
language and may not be used as operators.  Furthermore, operators
containing the comment seequences \texttt{/*} and \texttt{//} may not
be used as operators.


\section{Statements and Expressions}

\subsection{Exceptions}

Several locations in this document refer to an exception being thrown.
The semantics of an exception being thrown is that the application
halts with a run-time error.  Future versions of this document will
define facilities for propagating and catching exceptions.

\subsection{Statements}


\begin{quote}
\ntermdef{Stmt}

\defspace \nterm{Expr}

\defspace \nterm{VarDecl}

\defspace \nterm{StateValDecl}


\ntermdef{VarDecl}

\defspace \nterm{Specifier} \opt{\nterm{Type}} \nterm{Identifier} = \nterm{Expr}

\ntermdef{StateValDecl}

\defspace \keyw{stateval} \nterm{Identifier} \nterm{StateBinding}


\ntermdef{Specifier}

\defspace \keyw{val}

\defspace \keyw{var}

\end{quote}

Statements are either expressions, or variable declarations.  A
variable declaration must include an initial value.  Object variables are
declared with the \keyw{val} or \keyw{var} keyword; the former
indicates a final let binding, whereas the latter indicates a
assignable variable that can be updated.
State variables are declared with the \keyw{stateval} keyword. 

An optional type may be given for variable declarations.  If the type
is omitted for a \keyw{val} declaration, then it has the
structure of the initializing expression and the default permission
 for that structure. If no type is given for a \keyw{var} 
declaration, the variable is has type \keyw{dynamic}.

Statements evaluate to values, based on the expression in the
statement or the value of the initializer for the variable.  The last
statement in a sequence is used for the return value of a method or
the result of a block.

\subsection{Expressions}

\begin{quote}

\ntermdef{Expr}


\defspace \keyw{fn}   \opt{\nterm{MetaArgsSpec}}(\opt{\nterm{Args}})
              \opt{[\nterm{Args}]}
              => \nterm{Expr}

\defspace \nterm{Expr1}

\end{quote}

A first-class function includes standard polymorphic and parameter
argument declarations. The optional arguments surrounded by [] specify
environment variables that are captured by the lambda at its declaration.
The syntax of polymorphic parameters are described on
Section \ref{sec:polymorphic}. Polymorphic parameters specify which
data group permission a functions requires and allow the function to
be generic on which data groups it operates (similar to generic
methods on Java).
  
\begin{quote}

\ntermdef{Expr1}

\defspace \opt{\nterm{SimpleExpr} .} \nterm{Identifier} = \nterm{Expr}

\defspace \nterm{SimpleExpr} \texttt{<- }\nterm{State}

\defspace \nterm{SimpleExpr} \texttt{<{}<-} \nterm{State}

\defspace \keyw{atomic} \nterm{MetaArgs}  BlockExpr

\defspace \keyw{split} \nterm{MetaArgs}  BlockExpr

\defspace \keyw{unpack} BlockExpr

\defspace \keyw{match} ( \nterm{InfixExpr} ) \{ \seq{CaseClause} \}

\defspace \nterm{InfixExpr}

\end{quote}

The assignment form is for fields or for already-declared local
variables, which must have been declared using \keyw{var}. 

The state change operator \texttt{<-} modifies the object to the left of the
arrow as follows:

\begin{itemize}

\item
All tags on the right are added to the object.  Old tags are kept
unless they are inconsistent with the new tags, i.e. the old tag and
new tag are (transitively) different cases of the same state.

\item
All members that were declared in tags being removed, are removed from
the object.

\item
All members on the right are added to the object.  All old members on
the left that are not explicitly removed according to the bullet
above are retained.  

\item 
Futher details on the semantics of state change can be found in \citet{sunshine2011}.

\end{itemize}

The replacement operator \texttt{<{}<-} removes all tags and members from the object on the left and adds all tags and members of the state on the right.   

The type of either state change operations is \keyw{void}.

This paragraph provides a short overview of the \AE{}minium-specific
expressions. For a full definition of the semantics refer to
\cite{stork09:concurrency_by_default, stork10:uaeminium_spec}.  The
\keyw{atomic} expression provides a safe access environment to all
shared objects which belong to the data groups mentioned by the
\keyw{atomic} block. The \emph{MetaArgs} describes which data groups
the atomic block guarantees mutual exclusive access. Each
\emph{MetaArg} must refer to a valid data group (i.e., a concrete
declared data group or a group parameter).  The \keyw{split} executes
all statements of its body concurrently. To allow parallel access to
shared data the \keyw{split} block will split the declared data group
permissions into shared permissions, one for each statement.  
\keyw{unpack} is used to trade the group/access permission to the
specified object to gain access to the inner/nested groups declared
inside the object.

The \keyw{match} expression matches an input expression to one of
several cases using the \nterm{CaseClause} construct defined below.
The overall match expression evaluates to whatever value the
chosen case body evaluates to.

\begin{quote}

\ntermdef{CaseClause}

\defspace \keyw{case} \nterm{Pattern} \nterm{BlockExpr}

\defspace \keyw{default} \nterm{BlockExpr}

\ntermdef{Pattern}

\defspace \nterm{QualifiedIdentifier}


%\defspace \keyw{default}


\ntermdef{QualifiedIdentifier}

\defspace \nterm{Identifier} \seq{ . \nterm{Identifier}}

\end{quote}

The value is matched against each of the cases in order.  For the
first case that matches, the corresponding expression list is
evaluated.  If no pattern matches, an exception is
thrown.

A \keyw{case} tests the value's tags against
the \nterm{QualifiedIdentifier} given by the specified pattern.  The match succeeds if
one of the tags of the matched value is equal to the tag
\nterm{QualifiedIdentifier}, or if one of the tags of the matched value
was declared in a state that is a transitive \keyw{case of} the
\nterm{QualifiedIdentifier} specified.

The \nterm{QualifiedIdentifier}
must resolve to a state declared with the \keyw{state}
keyword; otherwise, an exception is
thrown. 

For the \keyw{default} case, the match always succeeds.  If
there is a \keyw{default} case, it must be the last one in the match
expression.



\begin{quote}

\ntermdef{InfixExpr}

\defspace \nterm{SimpleExpr}

\defspace \nterm{CastExpr}

\defspace \nterm{InfixExpr} \nterm{IdentifierOrOperator} \nterm{InfixExpr}


\ntermdef{IdentifierOrOperator}

\defspace \nterm{Identifier} \alt \nterm{Operator}


\ntermdef{CastExpr}

\defspace \nterm{SimpleExpr} \opt{\keyw{as} \nterm{Type}}

\end{quote}

The operators defined in Java have the same precedence in Plaid as
they do in Java, except the ternary operator and right shift operators, 
which are unsupported.  Identifiers as well as symbolic operators can be
used as infix operators; both are treated as method calls on the
object on the left of the operator.  Non-Java operators and
identifiers used as infix operators have a precedence above assignment
and state change, and below all other operators.

Cast expressions assert that a variable has a given type, and
also assert the relevant permission for that variable.  These casts
are trusted by the typechecker, but unchecked. A program that executes 
an invalid cast may fail at any point later in the program's 
execution.

\begin{quote}

\ntermdef{SimpleExpr}

\defspace \nterm{BlockExpr}

\defspace \keyw{new} \nterm{State}

\defspace \nterm{SimpleExpr2}

\end{quote}

The \keyw{new} statement creates an object initialized according to the
\nterm{State} specification given (defined below).

\begin{quote}

\ntermdef{BlockExpr}

\defspace \{ \opt{\nterm{StmtListSemi}} \}

\ntermdef{StmtListSemi}

\defspace \nterm{Stmt} \seq{ ; \nterm{Stmt}} \opt{;}

\end{quote}

Block expressions have a semicolon-separated list of statements, with
an optional semicolon at the end.  The statement list evaluates to the
value given by the last statement in the list.

\begin{quote}

\ntermdef{SimpleExpr2}

\defspace \nterm{SimpleExpr1}

\defspace \nterm{SimpleExpr2} \nterm{BlockExpr}

\end{quote}

To enable control structures with a natural, Java-like syntax, we allow
a function to be invoked by passing a block expression as an argument. The block expression in this 
case is treated as a zero-argument, anonymous lambda.For example, the \keyw{if} construct is defined in 
Plaid Standard Library so one can write \code{\keyw{if}(\emph{boolean value})\{...\};} in Plaid. One 
difference from Java syntax is that all expressions must be followed by a semi-colon, so \keyw{if} 
and \keyw{while} in Plaid must have trailing semi-colons unlike in Java.


\begin{quote}

\ntermdef{SimpleExpr1}

\defspace \nterm{Literal}

\defspace \nterm{Identifier}

\defspace \keyw{this}

\defspace ( \nterm{Expr} ) 

\defspace \nterm{SimpleExpr1} . \nterm{Identifier}

\defspace \nterm{SimpleExpr1} . \keyw{new}

\defspace \nterm{SimpleExpr1}  \opt{\nterm{MetaArgs}} \nterm{ArgumentExpr}

\ntermdef{ExprList}

\defspace \nterm{Expr} \seq{ , \nterm{Expr}}


\ntermdef{ArgumentExpr}


\defspace (  \opt{\nterm{ExprList}} ) 

\end{quote}

\keyw{this} represents the receiver of a method call as in Java and is 
bound in method bodies declared as members of states.  Unlike Java,
\keyw{this} is not bound in field initializers.

Expressions can appear within parentheses as a comma
separated list representing a tuple.

Java constructors can be invoked by calling \keyw{new}
on the Java class name.

Function and method invocation are handled uniformly by
supplying the arguments as a tuple.  Applications can be
chained, supporting currying.  Polymorphic arguments
are specified at each call site as well.

\section{Polymorphism}
\label{sec:polymorphic}

\begin{quote}

\ntermdef{MetaArgsSpec}

\defspace < \nterm{MetaArgSpec} \opt{, \nterm{MetaArgSpec}} >

\ntermdef{MetaArgSpec}

\defspace \keyw{group} \opt{\nterm{GroupPermission}} \nterm{Identifier}

\ntermdef{GroupPermission}

\defspace \keyw{exclusive}

\defspace \keyw{shared}

\defspace \keyw{protected}

\ntermdef{MetaArgs}

\defspace < \nterm{SimpleExpr1} \opt{, \nterm{SimpleExpr1}} >

\end{quote}

Plaid supports polymorphism for data groups\footnote{Extending
  polymorphism to types should be straightforward.} and uses angle
brackets to enclose polymorphic parameters and arguments (similar to
Java's generics). A \emph{MetaArgSpec} describes a single
 polymorphic formal parameter. At the moment Paid only supports only
group parameters. A group parameter consists of the 
keyword \keyw{group} to identify this parameter as group parameter, and optional
\emph{GroupPermission} (only for state declarations) and the
name of the parameter. For more information about data groups and
group parameters refer to \cite{stork09:concurrency_by_default,
  stork10:uaeminium_spec}.

\section{Declarations}

\begin{quote}
\ntermdef{Decl}

\defspace \seq{\nterm{ModifierOrDefaultPermission}} \keyw{state} \nterm{Identifier}  \opt{\nterm{MetaArgs}} \\
          \indent~~~~~~~~~~~~~~~~~~~\opt{\keyw{case} \keyw{of} \nterm{QualifiedIdentifier}  \opt{\nterm{MetaArgs}}}
          	\opt{\nterm{StateBinding}} \opt{;}

\defspace  \seq{\nterm{ModifierOrDefaultPermission}} \keyw{stateval} \nterm{Identifier}  \opt{\nterm{MetaArgs}} \\
           \indent~~~~~~~~~~~~~~~~~~~\opt{\nterm{StateBinding}} \opt{;}

\defspace \seq{\nterm{Modifier}} \nterm{MSpec} ;

\defspace \seq{\nterm{Modifier}} \nterm{MSpec} \nterm{BlockExpr}

\defspace \seq{\nterm{Modifier}} \nterm{FieldDecl} ;

\defspace \seq{\nterm{Modifier}} \nterm{GroupDecl} ;




\end{quote}

\keyw{state} and \keyw{stateval} declarations specify the implementation of a state,
as specified in the state definition. The \keyw{state} keyword means that this state is given its
own \textit{tag} that can be used to test whether objects are in that state.  Only states declared with \keyw{state} can be given in a pattern for a case in a \keyw{match} statement.

The \keyw{case} \keyw{of} keyword sequence assigns a superstate. States have 
all of the members of a superstate. Different cases of the same superstate 
are orthogonal; no object may ever be tagged with two cases of the same superstate.

The final two declarations are for method and field declarations.  The
method declaration has a method header
and an optional method body.  If the body is missing then
the method is abstract and must be filled in by sub-states or when the
state is instantiated.

Fields and state operators are discussed in more detail below.

\pagebreak

\begin{quote}

\ntermdef{StateBinding}

\defspace = \nterm{State}

\defspace \{ \seq{\nterm{Decl}} \}

\end{quote}

\begin{quote}

\ntermdef{State}

\defspace \nterm{StatePrim} \seq{\keyw{with} \nterm{StatePrim}} % left-associative

\ntermdef{StatePrim}

\defspace \{ \seq{\nterm{Decl}} \}

\defspace \nterm{SimpleExpr1} \opt{\{ \seq{\nterm{DecOrStateOp}} \}}

\defspace \keyw{freeze} \nterm{SimpleExpr1}

\ntermdef{DeclOrStateOp}

\defspace \nterm{Decl}

\defspace \nterm{StateOp}

\ntermdef{StateOp}

\defspace \keyw{remove} \nterm{Identifier} ;

\defspace \keyw{rename} \nterm{Identifier} \keyw{as} \nterm{Identifier} ;

\end{quote}

A \nterm{StateBinding} can either be an assignment to a \nterm{State} or
a list of declarations in curly braces.

A \nterm{State} is a composition of primitive states separated by the
\keyw{with} keyword.   The \nterm{StatePrim} category includes a list
of declarations, or an expression, that should evaluate to a state,
modified by a list of declarations and state operations (\nterm{DeclOrStateOp}).  
Expressions that evaluation to an object can be transformed into a primitive state
with the \keyw{freeze} keyword. 

Composition is in general symmetric, as in traits.  It is an error if two states are composed with a member in common. 
The conflict can be resolved manually using state operators \keyw{remove} and \keyw{rename}, which
respectively discard or change the name of a member in a state. 

\begin{quote}


%%%%%%%%%%%%%%%%%% Fields and Methods %%%%%%%%%%%%%%%%%%%%%

\ntermdef{GroupDecl}

\defspace \keyw{group} \nterm{Identifier} = \keyw{new} \keyw{group} ;

\ntermdef{FieldDecl}

\defspace \nterm{ConcreteFieldDecl}

\defspace \nterm{AbstractFieldDecl}

\ntermdef{ConcreteFieldDecl}

\defspace \opt{\nterm{Specifier}} \opt{Type} \nterm{Identifier} = \nterm{Expr}

\ntermdef{AbstractFieldDecl}

\defspace \nterm{Specifier} \nterm{Identifier} 

\defspace \opt{Specifier} \nterm{Type} \nterm{Identifier}

\ntermdef{Args}

\defspace \opt{\nterm{ArgSpec}} \nterm{Identifier} \seq{ , \opt{\nterm{ArgSpec}} \nterm{Identifier}}

\ntermdef{ArgSpec}

\defspace \nterm{Type} \opt{$>>$ \nterm{Type}}

\ntermdef{MSpec}

\defspace \keyw{method} \opt{\nterm{Type}} \nterm{IdentifierOrOperator} \opt{\nterm{MetaArgsSpec}} ( \opt{\nterm{Args}} )
          \opt{[ \nterm{Args} ]}

\end{quote}

Declaring a new data group is realized via \nterm{GroupDecl}, where identifier is the name of new 
data group. Data groups need to be immediately initialized with a new data group.$\!$\footnote{Future 
version might allow the declaration of data groups without initialization to realize data group 
parameters within structural types.} 

The \nterm{FieldDecl} form should be familiar from Java-like
languages.  If no expression is given then the field is abstract.  
When fields are first defined a specifier (\keyw{var} or \keyw{val})
must be given; later, when the field is overridden and given a concrete
value, the specifier may be omitted.
\keyw{var} fields are assignable, \keyw{val} fields
are not.

If a type is missing and an expression is given for a
  \keyw{val} field, then the type of the field is inferred from the
  expression as in variable declaration statements.  
  If the type is missing and either no expression is
  given or it is a \keyw{var} field, then the type is \keyw{dynamic}.

Method headers include specifications for their arguments.
Each argument specification includes the required type at the time of
the method call or function application.  If the parameter ends the
call with a different type this is indicated with a $>>$ and the
resulting type. If no resulting type is specified then it defaults to
the required type.  If no argument specification is given for
a variable, then its starting and ending type defaults to \keyw{dynamic}.

The method header \nterm{MSpec} has a standard format (similar to
functions).  As in function declarations, programmers may optionally
include types for any captured environment variables within square brackets. For methods
declared within states, the distinguished variable \keyw{this},
representing the receiver of the method, may appear in this list.  If
it does not, then the default specification for \keyw{this} is to have a required
and resulting type of the structure representing the state the method is defined in
with the default permission for that state as described below.

The name of an method can be an operator, which supports a simple form of \emph{operator overloading}. 
Operator methods are semantically equivalent to other methods. They dispatch only on the receiver, so 
\code{x + y} calls the \code{+} method defined on \code{x} with \code{y} as an argument.

\begin{quote}

%%%%%%%%%%%%%%%%%% Modifiers %%%%%%%%%%%%%%%%%%%%%

\ntermdef{Modifier}

\defspace \keyw{override}

\defspace \keyw{requires}

\ntermdef{DefaultPermission}

\defspace \keyw{immutable}

\ntermdef{ModifierOrDefaultPermission}

\defspace \nterm{Modifier}

\defspace \nterm{DefaultPermission}

\end{quote}


\keyw{override} indicates that a method overrides a function of the
same name during composition.

\keyw{requires} is similar to \keyw{abstract} in Java.  However,
things are more interesting in Plaid, because one can pass around an
object that has abstract/required members.  It is not necessary to
use the \keyw{requires} modifier in state definitions; one can simply
leave off the definition of a function.  \keyw{requires} is necessary
in types, however, to distinguish the presence vs. absence of a
member in that type.  Unlike in Java, methods may be called on an
object that has a required member, but only if the type given to the
method's receiver does not expect that member to be present.

A state with a default permission of \keyw{immutable} means that the permission of any fields,
local variables, or parameters declared to have the structure represented
by the state defaults to \keyw{immutable} when a permission
is not specified.  If the state is not give the default \keyw{immutable}
then the default permission is \keyw{unique}.




\section{Types}


\pII{
\begin{quote}
%
%%%%%%%%%%%%%%%%%% State and Declaration Types %%%%%%%%%%%%%%%%%%%%%
%
\ntermdef{StateTypeBinding}
%
\defspace <: \nterm{StateType}
%
\defspace = \nterm{StateType}
%
%
\ntermdef{StateType}
%
\defspace \nterm{StateTypePrim} \seq{\keyw{with} \nterm{StateTypePrim}} % left-associative
%
\ntermdef{StateTypePrim}
%
\defspace \nterm{QualifiedIdentifier} \opt{<\nterm{PermBindingList}>} \opt{\{ \nterm{RenameList} \}}
%
\defspace \{ \seq{\nterm{DeclType}} \}
%
\defspace \keyw{requires} \nterm{StateTypePrim}
%
\end{quote}
}

\pII{Interface definitions are analogous to state definitions, except that
the primitive declaration block gives only declaration types.  In
addition, interfaces can require whole other interfaces or blocks;
this is semantically equivalent to requiring each member in the
specified required interface.
%
\begin{quote}
%
\ntermdef{DeclType}
%
\defspace \seq{\nterm{Modifier}} \keyw{state} \nterm{Identifier}
          \opt{\keyw{case} \keyw{of} \nterm{QualifiedIdentifier}}
          \opt{\nterm{StateTypeBinding}}
%
\defspace \seq{\nterm{Modifier}} \keyw{interface} \nterm{Identifier}
          \opt{\keyw{case} \keyw{of} \nterm{QualifiedIdentifier}}
          \opt{\nterm{StateTypeBinding}}
%
\defspace \seq{\nterm{Modifier}} \keyw{type} \nterm{Identifier}
          \opt{\nterm{TypeDef}}
%
\defspace \seq{\nterm{Modifier}} \seq{\nterm{MSpec}}
%
\defspace \seq{\nterm{Modifier}} \nterm{Type} \nterm{Identifier}
%
\end{quote}
%
These are analogous to full declarations, except that states cannot be
given definitions, only types; methods cannot be given bodies; and
fields cannot be given initial values.
}

%%%%%%%%%%%%%%%%%% Typedefs %%%%%%%%%%%%%%%%%%%%%

\begin{quote}

\pII{
\ntermdef{TypeDef}
%
\defspace = \nterm{Type}
%
\defspace <: \nterm{Type}
}

\ntermdef{Type}

\defspace \nterm{ArgSpecs} \opt{[ \nterm{Args}]} -> \nterm{Type}

\defspace \nterm{QualifiedIdentifier}

\defspace \opt{\nterm{PermKind}} \nterm{State}\pII{\opt{@\nterm{State}}}

\defspace \keyw{none}

\defspace \keyw{dynamic}

\defspace (\nterm{Type})


\ntermdef{PermKind}

\defspace \keyw{unique}

\defspace \keyw{full}

\defspace \keyw{shared}

\defspace \keyw{pure}

\defspace \keyw{immutable}

\vII{\ntermdef{ArgSpecs}}

\vII{\defspace \nterm{ArgSpec} \seq{ * \nterm{ArgSpec} }}



\end{quote}

\pII{Type bindings can be either a definition or a lower bound, just like
interface and state bindings.  Types can be defined as a function
type, a reference to another type definition, or a combination of
permission and state.  It is also possible to have no permission at
all.}

Function types include optional arguments in brackets [] that specify
permissions to objects in scope, along with their state transitions.
\nterm{ArgSpecs} include just the permission and the state, while
\nterm{Args} includes the variable name as well (since for the optional
arguments we are naming variables currently in scope). The arrow operator (->) is right associative. Left associativity is supported by enclosing a \nterm{Type} in parentheses.

\vII{Formally, a function that accepts multiple arguments actually accepts an
argument tuple, which is written with a \code{*}-separated list.}


State types include an optional permission kind, which defaults to the permission kind of the state, or \keyw{dyn} otherwise.
\pII{The optional @ state is for \keyw{full}, \keyw{shared}, and \keyw{pure}, for which the first state given
is the state guarantee and the second state given is the current state,
which defaults to the state guarantee if missing.}

%Permission kinds are the 5 kinds from the Plural system, with the
%semantics described there.  \TODO{re-summarize}




\section{Compilation Units}

\begin{quote}

\ntermdef{CompilationUnit}

\defspace \keyw{package} \nterm{QualifiedIdentifier} \code{;} \nterm{Decls}

\end{quote}

A compilation unit is made up of a (required) package clause followed
by a sequence of declarations.

\pII{
\subsection{New Import Design (overrides below)}
%
Goals
 * modules should be parametric in their imports by default
%
 * programming in an extensible world should look like, and be no more
 costly than, programming in the Java style (packages and imports)
%
 * if you have security constraints and don't want things to be overridden,
there should be a (not too painful) way to enforce that
%
the import statement looks like Java, but means requires a module with
the right signature.
%
by default, an import gets resolved to the actual implementation module
named.
%
there is a mechanism for redirecting the import to some other module
that conforms to the signature (this is done in a surrounding module
or some kind of package mechanism)
%
there is a mechanism for sealing imports (really just binding them to
their defaults) to a module or to a package so they cannot be rebound
externally.
}
\subsection{Imports}

\begin{quote}

\ntermdef{Decl}

\defspace \keyw{import} \nterm{QualifiedIdentifier} \opt{\nterm{DotStar}} \code{;}

\ntermdef{DotStar}

\defspace \code{. *}

\end{quote}

An import statement imports a qualified name into the current scope so
it can be referred to by the last identifier in the qualified name.
If the import ends in .*, then all the members of the given
\nterm{Name} are imported into the current scope.  When a name from a
wildcard import is used, the compiler must be able to find the actual
declarations in that scope in order to properly resolve them; if the
declarations are unavailable, each declaration used must be imported
separately.  Any top-level external identifier used in the file must
be declared as an import, except for the identifier ``plaid'' for the
Plaid standard libraries; this preserves the property that the
compiler always knows what top-level identifiers are legal to access.

As in Java, importing the same simple name twice is an error unless
the fully qualified name is the same.  Importing a specific simple
name always overrides importing all elements of a package where
that name is defined, regardless of which definition goes first.
In general, Plaid follows the Java Language Specification section
7.5.

\TODO{Need to define imports more precisely!  look at Java.  Address
conflicts between local names and imports.}

\subsection{Copmilation Unit Semantics}

We define the semantics of a compilation unit in terms of ordinary
Plaid objects.  Each compilation unit is semantically an immutable
object with an ``instantiate()'' method that can be used to
instantiate a fresh module object.

Freshly generated module objects have methods defined for each of the
top-level method declarations in the compilation unit; however, top
level field declarations from the module are still missing.  They also
have abstract fields for the top levels of imported modules, which are
given the types declared in the source code~\footnote{There is no
  syntax for this yet}.  If no type for the import is given, the
compiler looks up the imported module according to its current
configuration\footnote{When targetting Java this is typically done
  with the PLAIDPATH, as defined below} and uses the type of the
module found for this import.  Note that even though the type of a
module found by the compiler may be used, the actual import is not
bound until the module is linked.

\cut{
It has a concrete field for the top level of the package being
created, and so forth with concrete fields going up to an object
representing the file.  In that object, imported names are replaced
with fully-qualified names.
%
The module object must be immutable; when instantiated, it executes no
code and allocates no storage (nor has storage allocated for it).
%
QUESTION: is this realistic?  Would it help if everyone had implicit
access to a ``system'' object?  Would it help if we made exceptions
only for state that is innocuous, e.g. caching, logging, or unique
token generation?
}

Code within the module object may refer to imported modules.  These
references are references to the abstract fields provided by the
imports.  In the case of an import m.* declaration, m is imported and
all names in the module that were found in m.* have an ``m''
prepended.

A module is linked to other modules by using the m <- { p = p\_def }
syntax, where p is the name of the imported module and p\_def is the
definition of the imported module, as specified at link time.  Linking
typically occurs whenever a module is first loaded into the run-time
system.

A fully-linked module object contains a special method
``initializeModule()'' which creates the proper fields and invokes
their initialization code.  At this point, the methods defined in the
module object can be invoked and the fields defined can be accessed.


\subsection{Module Composition and Hierarchy}

Two or more module objects may be composed using Plaid's \keyw{with}
operation to form a single module that has a union of the declarations
from each constituent module.  Plaid's composition semantics will
naturally merge imports with the same name from the two module
objects, but renaming can be used to distinguish them if desired.  The
module objects' \code{initializeModule} methods will conflict, so
these must be renamed and a new \code{initializeModule} method must be
defined, which should call the previous \code{initializeModule}
methods in order to properly initialize the constituent modules.

Modules may also be combined into a higher-level module, forming a
module hierarchy.  In this case, the higher-level module should be
created by calling plaid.system.modules.newModule(), and then the <-
operator should be used to add fields to the new module that refer to
each of the constituent modules.  Required fields should be defined
for imports of the higher-level module, and imports of the constituent
modules should be either linked to each other or should refer
\TODO{lazily} to the imports of the higher-level module.  The
\code{initializeModule} function should once again be defined to
properly initialize each of the constituent modules.

\TODO{actually, lets say they're initialized lazily.  How does the
runtime support this?  Do we check for bottom due to recursion?
does it help with code loading?}

Once a module has been linked or combined with another module, a new
package representing the combination may be formed by calling the
\code{asPackage} method.  This method returns a new package object,
such that calling ``instantiate()'' will generate fresh copies of the
module and everything it was linked with.  The method used to create
the package is passed the fully qualified name the package is to have.

Package objects support code generation appropriate to the platform,
allowing them to be written as binary code (or bytecode) in the file
system and loaded later.  \TODO{define the interface for this}


\TODO{define a convenient declarative linking system, in a tool spec}

\TODO{later, consider versioning, etc.}

\subsection{Package Loading}

Programmers can load package objects from within Plaid using a package
loader, using the code \code{plaid.system.defaultPackageLoader().loadPackage(\textit{$<$package-name$>$})}.
The default package loader looks up the module to be instantiated on
the PLAIDPATH, but other package loaders can\footnote{In the future,
  once we define the appropriate interface} be used to search for
modules in other ways.

\TODO{define the rest of the interface of PackageLoader}



\subsection{Applications}

An application is any globally-visible method that takes no arguments
and is declared at the top level of a package.

\noindent
The user can start an application using the following command line
syntax:

\begin{quote}
\cmdline{plaid packagename.functionname}
\end{quote}

\noindent
where \cmdline{packagename} is a potentially qualified name.  The
Plaid runtime will look up the package referred to by the packagename
and load it.  The package will be instantiated and its imports will be
linked with modules that are looked up and instantiated in the same
way.  Then \code{initializeModule} will be invoked, followed by the
top-level function the user specified.  The Plaid runtime can also be
invoked with simply the name of a package, in which case the package's
\code{main} function will be invoked.

\subsection{File System Conventions}

A compilation unit is stored in a file with extension .plaid.
The file
must be stored in a directory that can be found using the PLAIDPATH
mechanism described below.

% For example, all $x_1$.$x_2$ package must be stored in
%\$CLASSPATH\$/$x_1$/$x_2$/.

Each compilation unit may have one top-level declaration that has the
same name as the file name (without the .plaid extension).  This
declaration is the only one that is visible from other files. All
other declarations in the file are local.

Other public declarations may be placed in a special file named
package.plaid. There may be one package.plaid file per package. All
declarations in this file are public.

\TODO{later: all decls are private except those specified as public.
  even later: integrate with a mechanism supporting signature
  ascription}


The Plaid compiler and runtime look up packages using the PLAIDPATH
mechanism.  The PLAIDPATH is a path specified in the same way as a
CLASSPATH in Java.  In particular, it is a sequence of files or
directory names separated with a system-specific file separator
character, which is ':' in unix and ';' in Windows.  When looking up a
qualified name, the PLAIDPATH is searched in order first of prefixes
of the name, and next in the order of files in the PLAIDPATH, for
matching .plaid files or binary or bytecode files.  If a Plaid file is
found with no corresponding binary or bytecode file, the Plaid file is
dynamically compiled.

\cut{
Plaid uses the Java classpath mechanism to find files.  When searching
for a definition for a qualified name $x_1$.$x_2 \ldots x_n$, where $x_1$ is not in
scope, the system will search for a directory under the classpath
named $x_1$ and then look for a file named $x_2$ there.  If the file is a
directory, the search proceeds with $x_3$ and so forth.
%
For each top-level declaration in the file, a Java class in the package declared
is created with the name of the top-level declaration.  The Java class
implementing a declaration is found at run time using Java's normal
classpath-based lookup mechanism.
}





\section{Java Language Binding}

This section defines the Java language binding to Plaid, specifing
how Plaid and Java code can interoperate.


\minisec{Accessing Java from Plaid.}  Any java package, class, or
class member can be referred to via a qualified name.  Imported
name(s) can include a package, class, or class member from Java.
Instances of a Java class C may be created by invoking C.new(...)  and
passing appropriate arguments for one of the constructors of class C.
A static method m of C may be invoked with the syntax C.m(...).  An
instance methods of a Java object o may be invoked with the syntax
o.m(...).  Arguments passed to calls of Java constructors and methods
may be Java objects.  Plaid integers, strings, and booleans are
converted to appropriate Java primitive, String, and numeric object
types (e.g. java.lang.Integer) depending on the declared type of the
method's formal parameters.  If a Java method takes an Object or
plaid.runtime.PlaidObject as an argument, then a Plaid object can be
passed to it, allowing Java code to access Plaid objects.

\minisec{Implementing Java Interfaces.}  When a Plaid object is first
created, or a state is defined, the \keyw{with} operator can be used
to compose a Java interface.  In that case, any \keyw{new}
expression that creates an object with that state will generate a
Plaid object that extends the appropriate Java interface.  The Plaid
object may then be passed to a Java method that takes the interface
type as an argument.  Methods of the interface that are invoked by
Java are converted into calls to Plaid methods of the same name and
arguments, as described immediately below.

\minisec{Accessing Plaid from Java.}  Java code may invoke methods of
Plaid objects when those objects implement Java interfaces, as
decribed above, or reflectively through the plaid.runtime.PlaidObject
interface.  When calling a Plaid method through this interface, Java
objects of type Integer, String, Booleans, and other numeric objects
are converted into the corresponding Plaid types.  PlaidObjects and
Java objects are passed through unchanged, and their methods may be
invoked from Plaid in the usual way described above.  The detailed
interface of plaid.runtime.PlaidObject is specified in the javadoc
for that interface.

The Plaid object representing a given Plaid package is accessible in
Java via the static method packagename.getPackage().  Note that this
only works for a top-level Plaid package; the package must have its
imports resolved before sub-modules can be extracted from the package.

As a convenience, a Java function corresponding to each top-level
function is created, which starts the Plaid runtime and executes the
function as if the \cmdline{plaid} command-line utility had been used.
Thus Plaid programs can be invoked exactly like Java programs using
the java command line tool, as long as the Plaid program files and the
Plaid runtime jar file is in the CLASSPATH.


\TODO{how are packages packed into jar files?}

\TODO{Clean interoperability between the null value in Java and
the Plaid type system}

\TODO{Do we want PLAIDPATH to be different from the Java CLASSPATH?
  Maybe the CLASSPATH is used if PLAIDPATH is not defined?}







\pII{
\subsection{IDE considerations}
%
IDEs should show state changes where they occur in code.  For example:
%
\begin{quote}
\code{file.close()       file>>ClosedFile}
\end{quote}
}


\bibliographystyle{plainnat}
\bibliography{spec}


\end{document}
